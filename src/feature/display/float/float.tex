\usepackage{rotating} % provides rotating for floats and boxes; based on the graphics package
% \usepackage{lscape,pdflscape} % allows rotating multiple pages

% float typesetting
% \usepackage{float} % improved interface for floating objects, allows extra control on float objects; must be leaded before hyperref
% \usepackage{rotfloat} % links the float package and the rotating package
% \usepackage{floatflt} % float text around figures and tables which do not span the full width of a page; must use with the float package, and conflicts with the newfloat ecosystem
\usepackage{newfloat, keyfloat} % the keyfloat package provides mordern key-value grammar for new float types; conflicts with float and requires the newfloat package
% \usepackage{floatrow} %! conflicts with keyfloat
\usepackage{wrapfig} % substitude for the floatflt package
% \usepackage{endfloat} % puts all floating objects to the end of the document
\usepackage{placeins} % THOU SHALL NOT PASS!!! % provides barriers to force floats to be processed

\usepackage{minidocument} % allows displaying a shrinked copy of a document

\usepackage{qrcode}
\usepackage{hvqrurl}

% subfigure and caption object typesetting
%! as long as the driver is new enough, do not use the subfigure or subfig packages (subfigure especially)
\usepackage[font=small,labelsep=quad,indention=10pt]{caption} % big package; supersedes the legacy caption2 package; cf. \url{http://ctan.math.utah.edu/ctan/tex-archive/macros/latex/contrib/caption/caption-eng.pdf}
\usepackage[labelfont=bf,list=true]{subcaption}
\captionsetup[table]{textfont=it,position=top}
\captionsetup[subtable]{textfont=sf}
\captionsetup[figure]{position=bottom}
\captionsetup{labelsep=period}
%% above are the default values from the subcaption documentation \url{https://mirrors.concertpass.com/tex-archive/macros/latex/contrib/caption/subcaption.pdf}
\usepackage{bicaption} % bilingual captions; requires the babel and caption packages
\usepackage{mcaption} % provides an mcaption environment which puts figure or table captions in the margin
\usepackage{sidecap} % provides the SCfigure environment where the caption is put on the side
% \usepackage{subfig} %! legacy, do not use!
% \usepackage{fltpage} % provides the FPfigure and FPtable environments which allow the caption to be on the previous or next page than the figure itself %! not supported by Overleaf
\usepackage{sublabel}
% \usepackage{captcont} % allows enhanced control over the counter of captions %! legacy! do not use
\usepackage{minitoc} % moved here from toc.tex for compatibility
\usepackage{array} % do not use due to \url{https://tex.stackexchange.com/questions/213088/amsmath-matrix-vs-loading-the-array-package}
\usepackage{dcolumn} % allows alignment at decimal points
\usepackage{hhline}

\usepackage{supertabular} % jolly old many-pages table support
\usepackage{longtable} % a superseder; but have a few more problems. do not use this inside a multicol, and you are basically good to go

% \usepackage{slashbox} % needs direct attention
\usepackage{multirow} % the multicol package is in layout.tex and is not for tables
\usepackage{bigstrut}
\usepackage{bigdelim}
\usepackage{makecell}

\usepackage{tabularx}
\usepackage{xltabular}

% colorful tables
\usepackage{tcolorbox}
\usepackage{colortbl}
% the xcolor package also have related features

% \usepackage{mdwtab} % This reimplementation of the tabular and array environments re-does what the array package does, but in a way that builds on a reimplementation of the tabular and array functionality within the LATEX kernel itself. Among many improvements claimed for the package is that there are no built-in column types: all column types are explicitly declared within the package. 
%! This, though very neat, introduces serious incompatibility with the aforementioned packages.
\usepackage[inline, shortlabels]{enumitem} % powerful stuff. cf. \url{http://mirrors.ibiblio.org/CTAN/macros/latex/contrib/enumitem/enumitem.pdf}
% \usepackage{paralist} % functionality is now fully covered by the enumitem package
\usepackage{outlines}
% misc
\usepackage{schedule}
\usepackage{easy-todo}